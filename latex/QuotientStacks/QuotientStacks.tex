\documentclass[a4paper,11pt]{article}
\pagenumbering{arabic}
\usepackage{../header}
\usepackage{../quiver}

\begin{document}

\section{Quotient Stacks} 
We want to find out how stacks arise when taking quotients by group actions.
To start things off, let's recall what a group action of presheafs is. 
\begin{defi}[Group action on a presheaf.]
    Let $\cC$ be a (locally small) category, let $X: \cC^\opp \to \Sets$ be a 
    presheaf of sets and let $G: \cC^\opp \to \Grp$ be a presheaf of groups. 
    A \textit{left action} of $G$ on $X$ is a morphism $\sigma: G \times X \to X$
    which behaves like a left group action on points. In this case, $X$ is sometimes
    called a \textit{presheaf of $G$-sets.}
\end{defi}

\begin{defi}[Quotient presheaf]
    Given a presheaf of groups $G$ and a sheaf of $G$-sets $X$, we can build the
    quotient presheaf $(X/G)$, via the functor
    \begin{equation*}
        S \mapsto X(S)/G(S).
    \end{equation*}
\end{defi}

Instead of quotient sets, we will look at quotient groupoids.
\begin{defi}[Quotient Groupoid]
    Let $X$ be a set with a group action $G \lacts X$. Then we define
    the \textit{quotient groupoid} $[X/G]$ given by the category with
    objects $x \in X$ and $\Hom(x,x') = \{g \in G \mid gx = x'\}$. 
\end{defi}
One easily verifies (Exercise 0.4.7) that an action $G \lacts X$ is free (i.e.,
has trivial stabilizers) iff $[X/G]$ is a set.

A Quotient prestack is a scheme-theoretic/relative version of the quotient groupoid.
It behaves to quotient groupoids the same way quotient presheaves behave to 
quotient sets.

Also, let's switch to scheme world. From now on, $\cC = \SchOver S$. 

\begin{defi}[Quotient prestack]
Let $G \to S$ be a smooth affine group scheme acting on a scheme $U$ 
over $S$. We define the \textit{quotient prestack $[U/G]^\pre$} as the
category over $\SchOver S$ where the fiber category over an 
$S$-scheme $T$ is the quotient groupoid $[U(T)/G(T)]$. In other words,
the objects of $[U/G]$ are given by morphisms $T \to U$ over $S$.
A morphism from $u' \in U(T')$ to $u \in U(T)$ is the
data of a map $f: T' \to T$ and an element $g \in G(T')$ such that 
$u' = g \cdot (u \circ f)$. Composition of maps is given by 
$(g, f) \circ (g', f') = (g'g, f \circ f')$. 
\end{defi}
\exercise {2.3.15 a)} $[U/G]^\pre$ is a prestack. There is not much to check. 
It is a category fibered over groupoids by definition, which is already a good sign.
\begin{enumerate}
    \item \textit{Pullbacks exist.} Given an object $(a \mapsto W) \in
        [U/G]^{\pre}$ (which is really just a morphism $W \to U$) and a
        morphism $f: W' \to W$, we simply set $f^* a = (a \circ f)$. 
    \item \textit{Pullbacks are unique up to unique isomorphism.} This is a bit more
        involved, but the following diagram essentially gives the proof.
    \[\begin{tikzcd}[ampersand replacement=\&]
    	b \&\& {a'} \&\& a \\
    	\\
    	V \&\& {W'} \&\& W
    	\arrow[maps to, from=1-5, to=3-5]
    	\arrow[maps to, from=1-3, to=3-3]
    	\arrow["{(g_1,f)}"', from=1-3, to=1-5]
    	\arrow["f"{description}, from=3-3, to=3-5]
    	%\arrow["\lrcorner"{anchor=center, pos=0.125}, draw=none, from=1-3, to=3-5]
    	\arrow["{f'}"{description}, from=3-1, to=3-3]
    	\arrow["{(g_2, f \circ f')}", curve={height=-12pt}, from=1-1, to=1-5]
    	\arrow[maps to, from=1-1, to=3-1]
    	\arrow["{(g_2 g_1^{-1}, f')}"', dashed, from=1-1, to=1-3]
    \end{tikzcd}\]
        Here, we start with an arrow $a' \to a$, which comes with an element 
        $g_1 \in G(W)$ such that $a' = g \cdot (a \circ f)$. Similarly, an arrow 
        $b \to a$ comes with an element $g_2 \in G(W)$ such that 
        $b = g_2 \cdot (a \circ f \circ f')$. But now $b \to a$ factors through 
        $a' \to a$, as
        \begin{equation*}
            b = g_2 ( a \circ f \circ f') = g_2 g_1^{-1} \cdot g_1 \cdot (
            a \circ f) \circ f' = g_2 g_1^{-1} \cdot (a' \circ f'),
        \end{equation*}
        and this factorization is unique, as $(hg_1 = g_2 \iff h = g_2 g_1^{-1})$.
\end{enumerate}
Note that if $G$ acts freely on $U$ (i.e., the action map $(\sigma, p_2): 
G \times_S U \to U \times_S U$ is a monomorphism) if and only if 
$[U/G]^\pre$ is equivalent to a presheaf. Indeed, in this case, the 
action maps are monomorphisms on points. Hence we have $[U(T)/G(T)] \cong 
(U(T)/G(T))$ for all $T$, as there are no non-trivial stabilizers. This is
\exercise{2.3.19}.

Instead of thinking of an object of $[U/G]^\pre$ over a scheme $T$ 
as a morphism $f: T \to U$, we could also think of it as a trivial 
$G$-bundle over $T$:
\[\begin{tikzcd}[ampersand replacement=\&]
	{G \times T} \& U \\
	T,
	\arrow["{p_2}", from=1-1, to=2-1]
	\arrow["\tilde f", from=1-1, to=1-2]
\end{tikzcd}\]
where $\Tilde f$ sends $(g,s)$ to $g \cdot f(s)$. Hence, 
$[U/G]^\pre$ parametrizes the trivial $G$-bundles. In this setting,
it is clear that the quotient prestack is not a stack: If we are given 
a cover $\{T_i \to T\}$ of $T$ together with trivial $G$-bundles 
$G \times T_i \xto{\tilde f_i} X$ satisfying the cocycle conditions, the object 
we obtain from gluing these local data does not have to glue to a trivial
$G$-bundle again. Rather, this naturally leads to the definition of 
\textit{principal $G$-bundles}. 


\begin{defi}[Principal $G$-Bundle / Torsor]
    Let $G \to S$ be an fppf affine group scheme. A \textit{principal
    $G$-bundle} over an $S$-scheme $T$ is a 
    scheme $P$ with an action of $G$, given by a morphism
    $\sigma: G \times_S P \to P$, such that 
    $P \to T$ is a $G$-invariant\footnote{Remember that a morphism $P \to T$ is called $G$-invariant if the diagram
\[\begin{tikzcd}[ampersand replacement=\&]
	{G \times_S P} \&\& P \\
	P \&\& T
	\arrow["\sigma", from=1-1, to=1-3]
	\arrow[from=1-3, to=2-3]
	\arrow[from=2-1, to=2-3]
	\arrow["{p_2}"', from=1-1, to=2-1]
\end{tikzcd}\]
commutes.}
fppf morphism and 
    \begin{equation*}
        (\sigma, p_2): G \times_S P \to P \times_T P, \quad
        (g,p) \mapsto (gp, p)
    \end{equation*}
    is an isomorphism. (As isomorphisms can be checked fpqc-locally on $T$, this just
    says that the action $G \lacts U$ is faithfully free fpqc-locally. This 
    means that locally, $P$ is a trivial $G$-bundle.). If we equip $\SchOver S$
    with some Grothendieck-topology (or more generally, if everything lives on
    a site), we say that $P$ is a \textit{$G$-Torsor} if for every object $T
    \in \SchOver S$ there is a covering $\{T_i \to T\}$ such that $P(T_i) \neq
    \emptyset$. 
\end{defi}

    This motivates the definition of the quotient stack.
\begin{defi}[Quotient stack]
    We define the \textit{Quotient Stack} $[U/G]$ as the prestack over $\SchOver S$
    whose obects over an $S$ scheme $T$ are diagrams
\[\begin{tikzcd}[ampersand replacement=\&]
	P \&\& U \\
	\\
	T
	\arrow[from=1-1, to=3-1]
	\arrow[from=1-1, to=1-3]
\end{tikzcd}\]
where $P \to T$ is a principal $G$-bundle and $P \to U$ is a $G$-equivariant morphism
of schemes. A morphism $(P' \to T', P' \to U) \to (P \to T, P \to U)$ consists 
of a morphism $T' \to T$ and a $G$-equivariant morphism $P' \to P$ of schemes 
such that the diagram 
\[\begin{tikzcd}[ampersand replacement=\&]
	{P'} \&\& P \&\& U \\
	\\
	{T'} \&\& T
	\arrow[from=1-3, to=3-3]
	\arrow[from=1-3, to=1-5]
	\arrow[from=1-1, to=1-3]
	\arrow[from=1-1, to=3-1]
	\arrow[from=3-1, to=3-3]
	\arrow[curve={height=-18pt}, from=1-1, to=1-5]
	\arrow["\lrcorner"{anchor=center, pos=0.125}, draw=none, from=1-1, to=3-3]
\end{tikzcd}\]
is commutative and the left square is cartesian.
\end{defi}

\exercise {2.3.15 b)} $[U/G]$ is a prestack. This is straight forward. Pullbacks
are given by pullbacks, and one quickly verifies that indeed, if $P$ is a
principal $G$-bundle over $T$, $P \times_T T'$ is a principal $G$-bundle over
$T'$. The fact that pullbacks are universal just comes down to the fact that in
any (locally small) category, if we are given a square 
\[\begin{tikzcd}[ampersand replacement=\&]
	{P''} \&\& {P'} \&\& P \\
	\\
	{T''} \&\& {T'} \&\& T
	\arrow[from=1-5, to=3-5]
	\arrow[from=1-3, to=1-5]
	\arrow[from=1-3, to=3-3]
	\arrow[from=3-3, to=3-5]
	\arrow[from=3-1, to=3-3]
	\arrow[from=1-1, to=1-3]
	\arrow[from=1-1, to=3-1]
\end{tikzcd}\]
in which the right square is cartesian, the left square is cartesian iff the 
composition square is.

\subsection*{Interesting exercises} % (fold)
\exercise{2.3.22} \begin{enumerate}
    \item[(a)] Show that there is a morphism $p : U \to [U/G]^\pre$ and 
        a $2$-commutative diagram 
        \[\begin{tikzcd}[ampersand replacement=\&]
        	{G \times_S U} \&\& U \\
        	\\
        	U \&\& {[U/G]^\pre}
        	\arrow["p", from=1-3, to=3-3]
        	\arrow["p"', from=3-1, to=3-3]
        	\arrow["{p_2}"', from=1-1, to=3-1]
        	\arrow["\sigma", from=1-1, to=1-3]
            \arrow["\alpha"', shorten <=20pt, shorten >=20pt, Rightarrow,
            from=1-3, to=3-1]
        \end{tikzcd}\]
\item[(b)] The morphism $p$ is a categorical quotient among prestacks, i.e.
    for every $2$-commutative diagram
    \[\begin{tikzcd}[ampersand replacement=\&]
    	{G \times_S U} \&\& U \\
    	\\
    	U \&\& {[U/G]^\pre} \\
    	\&\&\& {\mathcal Z}
    	\arrow["p", from=1-3, to=3-3]
    	\arrow["p"', from=3-1, to=3-3]
    	\arrow["{p_2}"', from=1-1, to=3-1]
    	\arrow["\sigma", from=1-1, to=1-3]
        \arrow["\alpha"', shorten <=20pt, shorten >=20pt, Rightarrow, from=1-3,
        to=3-1]
        \arrow[""{name=0, anchor=center, inner sep=0}, "\varphi"',
        curve={height=12pt}, from=3-1, to=4-4]
        \arrow[""{name=1, anchor=center, inner sep=0}, "\varphi",
        curve={height=-12pt}, from=1-3, to=4-4]
        \arrow["\tau", shift left=5, curve={height=-6pt}, shorten <=17pt,
        shorten >=23pt, Rightarrow, from=1, to=0]
    \end{tikzcd}\]
    of prestacks, there exists a morphism $\chi: [U/G]^\pre \to \cZ$ and 
    $2$-isomorphisms compatible with $\alpha$ and $\tau$ such that everything 
    commutes. 
\end{enumerate}

\exercise{2.3.27} a) The square in part a) of the previous exercise is cartesian
(this should correspond to the fact that the group actions are simply transitive). 

\exercise{2.4.16} Let $G \to S$ be a smooth affine group scheme acting on 
a scheme $U$ over $S$. 
\begin{enumerate}
    \item[(a)] Show that $[U/G]^\pre$ satisfies Axiom $(1)$ of a stack over
        $\SchOver S_{\et}$. 
    \item[(b)] Show that $[U/G]$ is isomorphic to the stackification of 
        $[U/G]^\pre$ over $(\SchOver S)_{\et}$, and that 
        $[U/G]^\pre \to [U/G]$ is fully faithful.
\end{enumerate}

\exercise{2.4.17} Show that $U \to [U/G]$ is a categorical quotient among stacks. 

% subsection subsection name (end)

\contactend

\end{document}
